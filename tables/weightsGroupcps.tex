\begin{table}

\caption{\label{tab:rotweights:groups}Rotemberg weights for broad demographic groups.}
\centering
\resizebox{\linewidth}{!}{
\begin{threeparttable}
\begin{tabular}[t]{lllllllll}
\toprule
\cellcolor{gray!6}{shares} & \cellcolor{gray!6}{Women} & \cellcolor{gray!6}{Ages 18-24} & \cellcolor{gray!6}{Ages 25-34} & \cellcolor{gray!6}{Ages 35-49} & \cellcolor{gray!6}{Ages 50-64} & \cellcolor{gray!6}{Ages 65+} & \cellcolor{gray!6}{w/h.s.} & \cellcolor{gray!6}{college +}\\
CPS & 0.474 & 0.245 & 0.348 & 0.276 & 0.094 & 0.037 & 0.345 & 0.400\\
\cellcolor{gray!6}{Michigan} & \cellcolor{gray!6}{0.413} & \cellcolor{gray!6}{0.333} & \cellcolor{gray!6}{0.296} & \cellcolor{gray!6}{0.148} & \cellcolor{gray!6}{0.113} & \cellcolor{gray!6}{0.064} & \cellcolor{gray!6}{0.279} & \cellcolor{gray!6}{0.261}\\
\bottomrule
\end{tabular}
\begin{tablenotes}
\item \textit{Note: } 
\item Table reports fraction of overall Rotemberg weights attributable to a broader categorization of demographic groups. This table gives indication of which broad groups are key to identification.
\end{tablenotes}
\end{threeparttable}}
\end{table}
